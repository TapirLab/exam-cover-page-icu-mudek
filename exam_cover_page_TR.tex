%%%%%%%%%%%%%%%%%%%%%%%%%%%%%%%%%%%%%%%%%%%%%%%%%%%%%%%%%%%%%%%%%%%%%%%%%%%%%%%
%% Istanbul Commerce University Exam Cover Page (MÜDEK)
%% -------------------
%% $Author: Tapir Lab.$,
%% $Date: May 14th, 2022$,
%% $Revision: 1.1$
%% $Copyright: Tapir Lab.$
%%%%%%%%%%%%%%%%%%%%%%%%%%%%%%%%%%%%%%%%%%%%%%%%%%%%%%%%%%%%%%%%%%%%%%%%%%%%%%%

\documentclass[a4paper, 11pt]{article}
\setlength{\oddsidemargin}{0pt} 
\setlength{\evensidemargin}{0pt}
\usepackage[top = 0.98in, right = 0.98in, left = 0.98in, bottom = 0.98in]{geometry}

\usepackage[english]{babel}
\usepackage[utf8]{inputenc}
\usepackage[T1]{fontenc}

\usepackage{bm} % To type bold math

\usepackage{graphicx} % to include figures
\graphicspath{{./figures/}}

\usepackage{booktabs} % Required for tables
\usepackage{array} % extends the options for column formats
\usepackage{multicol} % to use multicolumns of tables


% Column Types
\newcommand{\PreserveBackslash}[1]{\let\temp=\\#1\let\\=\temp}
\newcolumntype{L}[1]{>{\PreserveBackslash\raggedright}m{#1}}
\newcolumntype{C}[1]{>{\PreserveBackslash\centering}m{#1}}

\begin{document}
	\boldmath
	\thispagestyle{empty}
	%%%%%%%%%%%%%%%%%%%%%%%%%%%%%%%%%%%%%%%%%%%%%%%%%%%%%%%%%%%%%%%%%%
	% University Logo - Department Name
	\begin{minipage}[c]{\textwidth}
		\begin{minipage}[c]{0.3\textwidth}
			\centering
			\includegraphics[width=\textwidth]{logo.png}
		\end{minipage}%
		\begin{minipage}[c]{0.7\textwidth}
			\centering
			{\LARGE \textbf{İstanbul Ticaret Üniversitesi}}\\[1em]
			{\Large \textbf{Elektrik--Elektronik Mühendisliği Bölümü}}
		\end{minipage}
	\end{minipage}
	
	%%%%%%%%%%%%%%%%%%%%%%%%%a%%%%%%%%%%%%%%%%%%%%%%%%%%%%%%%%%%%%%%%%%
	% Course code, name and type of exam
	\begin{center}
		\Large
		\textbf{EEE$405$--Digital Communication Systems (Final)}
	\end{center}
	
	%%%%%%%%%%%%%%%%%%%%%%%%%%%%%%%%%%%%%%%%%%%%%%%%%%%%%%%%%%%%%%%%%%
	% Student ID and name
	\begin{table}[!ht]
		\renewcommand{\arraystretch}{2}
		\centering
		\begin{tabular}{@{}|L{0.25\linewidth} C{0.05\linewidth} p{0.60\linewidth}|@{}}
			\hline
			\textbf{Öğrenci No}& : & ~ \\ \hline
			\textbf{Ad Soyad}  & : & ~ \\ \hline
		\end{tabular}
	\end{table}
	
	%%%%%%%%%%%%%%%%%%%%%%%%%%%%%%%%%%%%%%%%%%%%%%%%%%%%%%%%%%%%%%%%%%
	% Academic year, term, exam date, duration and academic personell
	% Exam "PÖÇ" information and grading table
	\begin{table}[!ht]
		\renewcommand{\arraystretch}{1.1}
		\centering
		\begin{tabular}{@{}>{\bfseries}L{0.18\linewidth}>{\bfseries}C{0.00\linewidth}>{\bfseries}L{0.25\linewidth}@{}}
			\toprule
			Akademik Yıl  & : & $2021$--$2022$                  \\ 
			\midrule
			Dönem         & : & \textbf{Bahar}                             \\ 
			\midrule
			Tarih         & : & $24$ Mayıs $2022$                           \\ 
			\midrule
			Süre          & : & $70$ \textbf{dk.}                \\
			\midrule
			Eğitmen & : & John Doe \\
			\bottomrule
		\end{tabular}
		\hfill
		\renewcommand{\arraystretch}{1.1}
		\begin{tabular}{>{\bfseries}C{0.1\linewidth}>{\bfseries}C{0.1\linewidth}>{\bfseries}C{0.1\linewidth}>{\bfseries}C{0.06\linewidth}}
			Soru & PÖÇ      & Ağırlık & Not \\
			\toprule
			$1$ & $1$.a, $1$.b & $35$    & ~   \\
			\midrule
			$2$ & $2$.a, $2$.b & $35$    & ~   \\
			\midrule
			$3$ & $1$.a, $2$.b & $30$    & ~   \\
			\midrule
			Toplam    & ~              		  & $100$   & ~   \\
			\bottomrule
		\end{tabular}
		\vspace{-0.3cm}
		\begin{flushright}
			\textbf{PÖÇ: Program Öğrenim Çıktıları}
		\end{flushright}
	\end{table}
	\vspace{-1cm}
	
	%%%%%%%%%%%%%%%%%%%%%%%%%%%%%%%%%%%%%%%%%%%%%%%%%%%%%%%%%%%%%%%%%%
	% Instructions
	\begin{center}
	\large
	\textbf{Sınav Kuralları}
	\end{center}
	\begin{itemize}
		\setlength{\itemsep}{0pt}
		\setlength{\parskip}{1pt}
		\item Sınava başlamadan önce isim ve numaranızı yazınız.
		\item \textbf{Tüm soruları cevaplayınız.} 
		\item Cevaplarınızı sınav kağıdınızda sorulardan sonra boş bırakılan yerlere yazınız. Yazınız temiz ve okunabilir olmalıdır. Cevabınızın anlamsız kısımları değerlendirilmeyecektir.
		\item Sınav esnasında sıranızda sadece sınav kağıdınız bulunmalıdır. Sıranız yazısız ve diğer tüm materyaller erişilemez konumda olmalıdır. Aşağıda listede verilen öğelerin bulundurulması yasaktır ve kopya çekme girişimi olarak değerlendirilir
		\begin{itemize}
			\setlength{\itemsep}{0pt}
			\setlength{\parskip}{1pt}
			\item[o] Sözlük
			\item[o] Hesap Makinesi
			\item[o] Cep Telefonu
			\item[o] Ders Notları/Sunumlar/Kişisel Notlar
			\item[o] Kitaplar
		\end{itemize}
	\end{itemize}
	
	\begin{table}[hb!]
		\centering
		\begin{tabular}{@{}C{0.03\linewidth}L{0.95\linewidth}@{}}
			\toprule
			\multicolumn{2}{l}{Program Öğrenim Çıktıları} \\
			\midrule
			$1$.a. & Matematik, fen bilimleri ve Elektrik-Elektronik Mühendisliğine özgü konularda yeterli bilgi birikimi. \\
			$1$.b. & Bu alanlardaki kuramsal ve uygulamalı bilgileri, karmaşık Elektrik-Elektronik Mühendisliği problemlerini kullanabilme becerisi. \\
			\midrule
			$2$.a. & Karmaşık Elektrik-Elektronik Mühendisliği problemlerini saptama, tanımlama, formüle etme ve çözme becerisi \\
			$2$.b. & Bu amaçla uygun analiz ve modelleme yöntemlerini seçme ve uygulama becerisi \\
			\bottomrule
		\end{tabular}
	\end{table}
	\unboldmath
\end{document}