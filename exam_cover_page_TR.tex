%%%%%%%%%%%%%%%%%%%%%%%%%%%%%%%%%%%%%%%%%%%%%%%%%%%%%%%%%%%%%%%%%%%%%%%%%%%%%%%
%% Istanbul Commerce University Exam Cover Page (MÜDEK)
%% -------------------
%% $Author: Tapir Lab.$,
%% $Date: May 14th, 2022$,
%% $Revision: 1.0$
%% $Copyright: Tapir Lab.$
%%%%%%%%%%%%%%%%%%%%%%%%%%%%%%%%%%%%%%%%%%%%%%%%%%%%%%%%%%%%%%%%%%%%%%%%%%%%%%%

\documentclass[a4paper, 11pt]{article}
\setlength{\oddsidemargin}{0pt} 
\setlength{\evensidemargin}{0pt}
\usepackage[top = 0.98in, right = 0.98in, left = 0.98in, bottom = 0.98in]{geometry}

\usepackage[turkish]{babel}
\usepackage[utf8]{inputenc}
\usepackage[T1]{fontenc}

\usepackage{bm} % To type bold math

\usepackage{graphicx} % to include figures
\graphicspath{{./figures/}}

\usepackage{booktabs} % Required for tables
\usepackage{array} % extends the options for column formats
\usepackage{multicol} % to use multicolumns of tables


% Column Types
\newcommand{\PreserveBackslash}[1]{\let\temp=\\#1\let\\=\temp}
\newcolumntype{L}[1]{>{\PreserveBackslash\raggedright}m{#1}}
\newcolumntype{C}[1]{>{\PreserveBackslash\centering}m{#1}}

\begin{document}
	\thispagestyle{empty}
	%%%%%%%%%%%%%%%%%%%%%%%%%%%%%%%%%%%%%%%%%%%%%%%%%%%%%%%%%%%%%%%%%%
	% Üniversite logosu ve bölüm ismi
	\begin{minipage}[c]{\textwidth}
		\begin{minipage}[c]{0.3\textwidth}
			\centering
			\shorthandoff{=} % trick for turkish and graphics inconsistency
			\includegraphics[width=\textwidth]{logo.png}
			\shorthandon{=} % trick for turkish and graphics inconsistency
		\end{minipage}%
		\begin{minipage}[c]{0.7\textwidth}
			\centering
			{\LARGE \textbf{İstanbul Ticaret Üniversitesi}}\\[1em]
			{\Large \textbf{Bilgisayar Mühendisliği Bölümü}}
		\end{minipage}
	\end{minipage}
	
	%%%%%%%%%%%%%%%%%%%%%%%%%%%%%%%%%%%%%%%%%%%%%%%%%%%%%%%%%%%%%%%%%%
	% Ders kodu, ders adı ve sınav türü
	\begin{center}
		\Large
		\textbf{BIL$\bm{452}$--Yapay Zeka (Final)}
	\end{center}
	
	%%%%%%%%%%%%%%%%%%%%%%%%%%%%%%%%%%%%%%%%%%%%%%%%%%%%%%%%%%%%%%%%%%
	% Öğrenci numarası ve ismi
	\begin{table}[!ht]
		\renewcommand{\arraystretch}{2}
		\centering
		\begin{tabular}{@{}|L{0.25\linewidth} C{0.05\linewidth} p{0.60\linewidth}|@{}}
			\hline
			\textbf{Öğrenci Numara}& : & ~ \\ \hline
			\textbf{Öğrenci İsim}  & : & ~ \\ \hline
		\end{tabular}
	\end{table}
	
	%%%%%%%%%%%%%%%%%%%%%%%%%%%%%%%%%%%%%%%%%%%%%%%%%%%%%%%%%%%%%%%%%%
	% Akademik yıl, dönem, sınav tarihi, sınav süresi, öğretim üyesi bilgileri
	% Sınav PÖÇ bilgileri ve notlandırması
	\begin{table}[!ht]
		\renewcommand{\arraystretch}{1.5}
		\centering
		\begin{tabular}{@{}|>{\bfseries}L{0.175\linewidth}>{\bfseries}C{0.00\linewidth}>{\bfseries}L{0.32\linewidth}|@{}}
			\hline
			Akademik Yıl  & : & $\bm{2017}$--$\bm{2018}$                     \\ \hline
			Dönem         & : & \textbf{Bahar}                     \\ \hline
			Tarih         & : & $\bm{28.05.2018}$                       \\ \hline
			Süre          & : & $\bm{75}$ \textbf{Dakika}               \\ \hline
			Öğretim Üyesi & : & \textbf{John Doe}\\ \hline
		\end{tabular}
		\hfill
		\renewcommand{\arraystretch}{1.1}
		\begin{tabular}{|>{\bfseries}C{0.08\linewidth}|>{\bfseries}C{0.09\linewidth}|>{\bfseries}C{0.07\linewidth}|>{\bfseries}C{0.04\linewidth}|}
			\hline
			Soru   & PÖÇ      & Ağırlık & Not \\ \hline
			$\bm{1}$ & $\bm{1}$.a, 1.b & $\bm{30}$    & ~   \\ \hline
			$\bm{2}$ & $\bm{2}$.a, 2.b & $\bm{15}$    & ~   \\ \hline
			$\bm{3}$ & $\bm{2}$.b      & $\bm{25}$    & ~   \\ \hline
			$\bm{4}$ & $\bm{2}$.a, 2.b & $\bm{30}$    & ~   \\ \hline
			Toplam & ~        & 100     & ~   \\ \hline
		\end{tabular}
		\vspace{-0.5cm}
		\begin{flushright}
			\textbf{PÖÇ: Program Öğrenim Çıktısı}
		\end{flushright}
	\end{table}
	\vspace{-1cm}
	
	%%%%%%%%%%%%%%%%%%%%%%%%%%%%%%%%%%%%%%%%%%%%%%%%%%%%%%%%%%%%%%%%%%
	% Sınav Kuralları
	\begin{center}
		\large
		\textbf{Sınav Kuralları}
	\end{center}
	\begin{itemize}
		\setlength{\itemsep}{0pt}
		\setlength{\parskip}{1pt}
		\item Sınava başlamadan önce isim ve numaranızı yazınız.
		\item \textbf{Tüm soruları cevaplayınız.} 
		\item Cevaplarınızı sınav kağıdınızda sorulardan sonra boş bırakılan yerlere yazınız. Yazınız temiz ve okunabilir olmalıdır. Cevabınızın anlamsız kısımları değerlendirilmeyecektir.
		\item Sınav esnasında sıranızda sadece sınav kağıdınız bulunmalıdır. Sıranız yazısız ve diğer tüm materyaller erişilemez konumda olmalıdır. Aşağıda listede verilen öğelerin bulundurulması yasaktır ve kopya çekme girişimi olarak değerlendirilir
		\begin{itemize}
			\setlength{\itemsep}{0pt}
			\setlength{\parskip}{1pt}
			\item[o] Sözlük
			\item[o] Hesap Makinesi
			\item[o] Cep Telefonu
			\item[o] Ders Notları/Sunumlar/Kişisel Notlar
			\item[o] Kitaplar
		\end{itemize}
	\end{itemize}
	
	\begin{table}[hb!]
		\centering
		\begin{tabular}{@{}C{0.03\linewidth}L{0.95\linewidth}@{}}
			\toprule
			\multicolumn{2}{l}{Program Öğrenim Çıktıları} \\
			\midrule
			$1$.a. & Matematik, fen bilimleri ve ilgili mühendislik disiplinine özgü konularda yeterli bilgi birikimi \\
			$1$.b. & Bu alanlardaki kuramsal ve uygulamalı bilgileri, karmaşık mühendislik problemlerinde kullanabilme becerisi. \\
			\midrule
			$2$.a. & Karmaşık mühendislik problemlerini saptama, tanımlama, formüle etme ve çözme becerisi \\
			$2$.b. & Bu amaçla uygun analiz ve modelleme yöntemlerini seçme ve uygulama becerisi. \\
			\bottomrule
		\end{tabular}
	\end{table}
\end{document}